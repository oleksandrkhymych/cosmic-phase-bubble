\documentclass[11pt,a4paper]{article}

\usepackage[margin=2.5cm]{geometry}
\usepackage{amsmath,amssymb}
\usepackage{hyperref}
\usepackage[T1]{fontenc}
\usepackage[utf8]{inputenc}

\title{Emergent Phase-Field Bubble Cosmology\\
  \large A phenomenological and exploratory framework}
\author{Oleksandr Khymych}
\date{2025-12-06}

\begin{document}
\maketitle

\begin{abstract}
We present an early-stage conceptual framework in which cosmological-scale
structures are interpreted as emergent phase-separated bubbles within a
continuous medium. Instead of postulating fundamental long-range forces, the
model explores whether local interaction rules, phase-field dynamics, and
effective surface tension can give rise to expanding, contracting, and merging
bubble-like domains. This work is intended as a toy model and exploratory
investigation rather than as a complete or experimentally validated theory.
\end{abstract}

\section{Introduction}

Modern cosmology usually describes the large-scale evolution of the universe
by combining general relativity with specified matter and energy components.
This approach is extremely successful, but it treats spacetime geometry and
long-range interactions as fundamental ingredients of the description.

Here we explore a different, phenomenological perspective: that cosmological
behaviour might be viewed as emerging from local interactions in a continuous
medium, in a way that is conceptually closer to hydrodynamics and phase
separation than to explicit force mediation. In this picture, large-scale
structures resemble bubbles or domains of a phase field, surrounded by other
phases of the same underlying medium.

\section{Conceptual framework: Emergent Stability Framework (ESF)}

In this work, we introduce what we refer to as the \emph{Emergent Stability
Framework (ESF)}. Within ESF, physical entities such as particles and extended
structures are not treated as fundamental objects, but as locally stable or
metastable configurations of an underlying continuous medium. The medium
itself is assumed to be continuous at the scales of interest, while its
microscopic nature is left unspecified.

In this interpretation, what we usually call a ``particle'' corresponds to a
pattern that the medium can sustain in a robust way. A configuration is
considered stable if small perturbations relax back to it, and metastable if
it persists for long but can eventually decay into other configurations.
Apparent motion of objects is then understood as the relocation of the regions
in which a given stable pattern is supported, mediated by local
reconfigurations of the medium rather than by the transport of a fixed
material entity.

Observable quantum phenomena---such as localisation, interference, and
entanglement---are interpreted as consequences of stability redistribution
under localised interactions, rather than as manifestations of fundamental
indeterminism or wavefunction collapse. A measurement corresponds, in this
language, to a process in which most metastable alternatives are destroyed by
a strong local interaction, leaving only one energetically favored pattern
that the medium can sustain in the presence of the measuring apparatus.

On cosmological scales, the same logic is applied to bubble-like domains of a
phase field: a ``universe'' is modeled as a large-scale stable or metastable
configuration (a bubble) within the medium, whose expansion, contraction, and
merger with other domains follow from the same stability-based principles.

\section{Local excitations in ESF: electrons and photons}

Within the Emergent Stability Framework, microscopic excitations such as
electrons and photons are described in terms of different types of stability
patterns supported by the medium.

\subsection{Electrons as locally stable configurations}

An electron is modeled as a locally stable configuration of the medium with
fixed integral properties (such as effective charge and mass), rather than as a
point-like object moving through space. Its persistence is understood as the
reproducibility of a particular pattern: when local conditions allow, the
medium relaxes into that configuration in some region, and what we call the
``same'' electron is the continued existence of this pattern, possibly at
different locations over time.

Apparent motion of an electron corresponds to a relocation of the region in
which this stable configuration exists, driven by local energy minimisation.
Under sufficiently strong interactions, the configuration may lose stability in
one region and reappear in another, with the underlying medium redistributing
energy and charge accordingly. From this perspective, an experiment that
``measures the position'' of an electron does not reveal a preexisting
trajectory, but selects one of the few regions where the medium can still
support the electron pattern after the interaction with the measuring device.

\subsection{Photons as propagating reconfiguration modes}

Photons, in contrast, are not associated with locally stable configurations.
Instead, they are modeled as propagating modes of stability redistribution in
the medium. A photon corresponds to a coherent, directed sequence of local
reconfigurations, in which loss of stability in one region is compensated by
the onset of a compatible configuration in a neighboring region, forming a
travelling pattern.

Because photons do not admit a rest state---there is no locally stable
``photon configuration'' that can remain static---their propagation speed is
set by the characteristic relaxation properties of the medium. In vacuum, this
leads to a universal propagation speed, while the wavelength of the photon
reflects the spatial scale over which the medium can coherently support a
given momentum transfer. Interaction with structured media does not change the
photon energy but modifies the allowed propagation modes through boundary
conditions, leading to refraction, reflection, or absorption.

When a photon interacts with an electron or with matter in general, the
propagating reconfiguration mode can terminate: the medium instead relaxes
into a locally stable configuration (such as an excited electronic state), and
the photon ceases to exist as a separate propagating pattern. In this view,
emission and absorption are transitions between different stability regimes of
the same underlying medium, rather than events involving the creation or
annihilation of a distinct material particle.

\section{Phase-field interpretation}

To describe cosmological-scale structures, we introduce a scalar phase field
$\phi(x,t)$ that encodes local states of the medium. Different stable phases
correspond to minima of an effective potential for $\phi$, while interface
regions arise from gradient-energy terms that penalise sharp transitions.

Qualitatively, the dynamics of $\phi$ are governed by local diffusion-like
processes, free-energy minimisation, and effective surface tension at phase
boundaries. Bubbles of one phase inside another arise as domains in which
$\phi$ is close to a particular minimum of the potential, separated by
interfaces where $\phi$ interpolates between distinct minima.

\section{Bubble dynamics and merging}

Within this framework, isolated bubbles tend toward approximately spherical
shapes due to surface-tension minimisation: interfaces evolve in a way that
reduces total surface area at fixed volume. When distinct bubbles approach and
interact, their interfaces can merge, leading to transient non-spherical
configurations and eventual relaxation toward new equilibrium shapes.

Such behaviour is familiar from classical fluid systems and motivates the
interpretation of large-scale cosmological events---such as expansion,
collision, and merging of domains---as manifestations of phase-field dynamics
rather than purely as motion in a fixed background spacetime.

\section{Relation to existing cosmology}

The present model is not intended to replace standard cosmological theory.
Instead, it offers a complementary phenomenological perspective through which
familiar concepts---expansion, contraction, and large-scale structure---may
be reinterpreted in terms of phase dynamics and emergent stability. Connections
to inflationary or cyclic cosmological scenarios are speculative at this stage
and are left for future exploration.

\section{Numerical exploration (preview)}

Preliminary numerical experiments based on phase-field simulations are under
active development. These simulations aim to visualise bubble formation and
merging, explore regimes of stability and instability, and identify generic
dynamical patterns that might serve as qualitative analogues of cosmological
behaviour. Details of numerical methods and code implementation are planned to
be documented separately in the accompanying simulation code.

\section{Limitations and open questions}

This framework is intentionally exploratory. Important open issues include the
lack of quantitative predictions, the absence of direct links to observational
data, and an unclear correspondence with relativistic spacetime dynamics.
Future work may clarify whether this phenomenological approach can serve as a
useful conceptual or computational tool, and whether ESF can be embedded into a
more complete theory that reproduces established results of quantum field
theory and general relativity.

\section{Draft interpretation: black holes in ESF}

Within the Emergent Stability Framework, black holes are interpreted not as
fundamental geometric singularities, but as regimes of the underlying medium
in which stable outward-propagating patterns can no longer be sustained.
This section presents a qualitative mapping intended to preserve consistency
with known phenomenology of general relativity, while remaining explicitly
exploratory.

In this interpretation, the event horizon corresponds to a phase boundary of
the medium. Outside the horizon, local excitations such as photons may exist
as stable propagating reconfiguration modes. As the boundary is approached,
the characteristic relaxation dynamics of the medium increasingly dominate
over outward redistribution of stability, leading to strong redshift and
effective time dilation for external observers.

Inside the horizon, any attempt to form outward-directed propagating modes
fails dynamically: perturbations are absorbed by the medium faster than they
can reorganize into stable patterns capable of escape. Information trapping
thus arises as a consequence of stability loss, rather than as a fundamental
causal prohibition.

Mergers of black holes are interpreted as boundary reconfiguration events
between adjacent high-density stability regimes. During such events,
non-spherical interface configurations relax toward new equilibrium shapes,
releasing excess interfacial stress as macroscopic propagating modes, in
qualitative correspondence with observed gravitational-wave signals.

Hawking radiation is treated as a boundary effect. Near the phase boundary,
microscopic fluctuations of the medium may occasionally organize into
outward-propagating stable modes. This endows the horizon with an effective
temperature determined by the curvature and scale of the boundary, without
invoking particle creation from the vacuum.

This interpretation is qualitative and phenomenological in nature.
Quantitative derivations, explicit connections to relativistic field equations,
and potential observational tests are left as subjects for future work.

\end{document}
